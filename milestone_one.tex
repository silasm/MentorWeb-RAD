\documentclass[11pt]{article}
\usepackage{makeidx}
\title{\textbf{MentorWeb\index{MentorWeb} Request Analysis Document (RAD)}}
\author{Joseph Richardson, Daniel Trebe,\\ Silas McCroskey, Halin Gordon,
Clifford Chan}
\makeindex

\begin{document}

\maketitle
\begin{center}
    Prepared for SE/CMPE 133: Software Engineering II, San Jose State Spring
    2014
\end{center}
\pagebreak

\tableofcontents
\setcounter{section}{-1}
\section{Revision History}
    \begin{tabular}{c|c|c|p{6 cm}}
        Revision & Originator  & Date              & Comments         \\ \hline
        1        & HammerSmash & \date{2014-02-17} & Initial Revision \\ \hline
        2        & HammerSmash & \date{2014-02-20} & Adding in requirements \\
        \hline
        3        & HammerSmash & \date{2014-02-24} & Adding Use Cases \\ \hline
        4        & HammerSmash & \date{2014-03-02} & Adding Sequence Diagrams,
                                                     use case models, and
                                                     non-fucntional requirements
                                                     \\ \hline
        5        & HammerSmash & \date{2014-03-04} & Moved into LaTeX file, added
                                                     Definitions, Acronyms, and
                                                     Abbreviations section
                                                     % TODO
    \end{tabular}

\section{Introduction}

    \subsection{Purpose}
        MentorWeb\index{MentorWeb} is a social media web-based application for
        working professionals.  MentorWeb\index{MentorWeb} provides a way for
        working professionals to develop a Mentor\index{Mentor}and
        Mentee relationship. Any user can be a mentor\index{Mentor}and a
        mentee\index{Mentee} at anytime. Users can follow mentors based on what
        the user's goals are matched to the mentor\index{Mentor}s background.

    \subsection{Scope}
        MentorWeb\index{MentorWeb} attempts to establish a stronger network for
        working professionals.\\
        \\
        MentorWeb\index{MentorWeb} will create a mentor\index{Mentor}and
        mentee\index{Mentee} relationship by allowing users who choose to sign
        up as a mentee\index{Mentee} to follow mentors\index{Mentor}.\\
        \\
        MentorWeb\index{MentorWeb} will suggest certain mentors for each
        mentee\index{Mentee} by matching the mentee's\index{Mentee} aspirations
        to the mentor's\index{Mentor} background.\\
        \\
        Each mentee\index{Mentee} will be able to communicate to their
        mentors\index{Mentor} in a variety of ways without requiring additional
        tools.

    \subsection{Objectives and Success Criteria}
        MentorWeb\index{MentorWeb} users will have a stronger professional
        network and professional foundation.

    % TODO
    \subsection{Definitions, Acronyms, and Abbreviations}
        \begin{tabular}{cp{10 cm}}
            Term                 & Definition \\ \hline
            Mentor\index{Mentor} & A professional with industry experience who
                                   volunteers to advise a mentee. \\ \hline
            Mentor\index{Mentee} & A student or worker new to the industry who
                                   seeks advising from a mentor. \\ \hline
            Sponsor\index{Sponsor}
                                 & A professional with industry experience who
                                   not only advises a Mentee, but also is
                                   involved in their career through sponsorship
                                   means, giving opportunities more directly
                                   \\ \hline
            MentorWeb\index{MentorWeb}
                                 & A web-based system which aims to facilitate
                                   the mentor\index{Mentor}-mentee\index{Mentee}
                                   relationships described above. \\ \hline
            The Program          & Used in this document where not otherwise
                                   obvious: the program MentorWeb. \\ \hline
            Shall                & When used in a requirement text, signifies
                                   that the program in question is incomplete or
                                   incorrect when not in compliance with this
                                   requirement. \\ \hline
            Should               & When used in a requirement text, signifies
                                   a desirable property of the target system,
                                   but one without which the system can still be
                                   considered complete and correct. \\ \hline
            API                  & Application Programming Interface \\ \hline
        \end{tabular}

    \subsection{References}
        IEEE Std 830-1998, IEEE Recommended Practice for Software Requirements
        Specifications

    \subsection{Overview}
        MentorWeb\index{MentorWeb} attempts to allow the user to establish a
        stronger professional network through facilitating
        mentor\index{Mentor}-mentee\index{Mentee} relationships.

\section{Current System}
    MentorWeb\index{MentorWeb} as a system has yet to be implemented; this
    document covers a new system which does not replace an existing one.

\section{Proposed System}
    This section describes the requirements and specifications of
    MentorWeb\index{MentorWeb}.

    \subsection{Overview}
        MentorWeb\index{MentorWeb} attempts to allow the user to establish a
        stronger professional network through a
        mentor\index{Mentor}-mentee\index{Mentee} relationship.

    \subsection{Functional Requirements}
    \begin{tabular}{|l|p{8 cm}|l|}
        \hline
        ID      & Description                             & Priority \\ \hline
        3.2.0.1 & Users shall be able to decide to keep
                  their conversations\index{Conversation}
                  off of server storage                   & EF 1 \\ \hline
        3.2.0.2 & MentorWeb shall allow
                  sponsors\index{Sponsor} with other
                  companies                               & EF 2 \\ \hline
        3.2.0.3 & Administrative users shall be able to
                  suspend users for abusing
                  MentorWeb\index{MentorWeb}              & EF 3 \\ \hline
        3.2.0.4 & In the event the user opts to deactivate
                  his or her account,
                  MentorWeb\index{MentorWeb} shall
                  support this, but shall not delete user
                  data to provide for the case that the
                  user opts to re-register                & EF 4 \\ \hline
        3.2.0.5 & MentorWeb\index{MentorWeb} shall be a
                  web-based project, and keep an open API
                  for expandability                       & EF 5 \\ \hline
                  3.2.0.6 & MentorWeb\index{MentorWeb}
                  should be designed to easily migrate to
                  a mobile application.                   & DF 1 \\ \hline
        3.2.1.1 & MentorWeb\index{MentorWeb} shall have a
                  login process and a registration process
                  for users                               & EF 6 \\ \hline
        3.2.1.2 & MentorWeb\index{MentorWeb} shall allow
                  login with the social media outlets
                  Facebook\index{Facebook} and
                  LinkedIn\index{LinkedIn}                & EF 7 \\ \hline
        3.2.2.1 & MentorWeb\index{MentorWeb} users shall
                  be able to registor as just a
                  mentor\index{Mentor}, just a
                  mentee\index{Mentee}, or both           & EF 8 \\ \hline
        3.2.2.2 & MentorWeb\index{MentorWeb} users shall
                  be able to input and display their
                  professional information (background,
                  aspirations, skills), and/or pull such
                  information from an existing
                  LinkedIn\index{LinkedIn} profile       & EF 9 \\ \hline
        3.2.2.3
    \end{tabular}


    % TODO
    \subsection{Nonfunctional Requirements}

        % TODO
        \subsubsection{Usability}

        % TODO
        \subsubsection{Reliability}

        % TODO
        \subsubsection{Performance}

        % TODO
        \subsubsection{Supportability}

        % TODO
        \subsubsection{Implementation}

        % TODO
        \subsubsection{Interface}

        % TODO
        \subsubsection{Packaging}

        % TODO
        \subsubsection{Legal}
    
    % TODO
    \subsection{System Models}

        % TODO
        \subsubsection{Scenarios}

        % TODO
        \subsubsection{Use Case Model}

        % TODO
        \subsubsection{Object Model}

        % TODO
        \subsubsection{Dynamic Model}

        % TODO
        \subsubsection{User Interface}

% TODO
\section{Glossary}

% TODO
\section{Appendices}

    % TODO
    \subsection{Hardware Requirements}

    % TODO
    \subsection{Project Plan}

    % TODO
    \subsection{Team Staffing and Responsibilities}

    % TODO
    \subsection{Notebook Log}

    % TODO
    \subsection{Index}
    \printindex
    

\end{document}
